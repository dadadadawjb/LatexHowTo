\section{Tabular}
\subsection{Tabbing}
\begin{tabbing}
    Form\hspace{3em}    \= Author \\
    Plain \TeX          \> Donald Ervin Knuth \\
    \LaTeX              \> Leslie Lamport
\end{tabbing}

\subsection{Tabular And Array}
\subsubsection{Overview}
\begin{tabular}[b]{l|cr|p{8em}}    % optional b for bottom, t for rop, no by default center
    left & center & right & fixed width \\
    \hline
    raggedright & centering & raggedleft & \centering This can automatically line break if it is longer than the fixed width \\
\end{tabular}
Ragged bottom.

Almost the same as tabular, not repeat.
\[
    \begin{array}{l|cr}
        \frac{1}{2} & 1 & 0 \\
        \hline
        0 & 1 & -\frac{1}{2} \\
    \end{array}
\]

\begin{tabular}{|c|*{3}{r@{.}l|}}
    \hline
    input & 12345&6 & 5000&0 & 1020&55 \\ \hline
    output & 765&43 & 5120&5 & 98760&0 \\ \hline
    net & 11580&17 & -120&5 & -97739&45 \\ \hline
\end{tabular}

\begin{tabular}{|c|*{3}{d|}}    % must usepackage dcolumn
    \hline
    name & \multicolumn{1}{c|}{wjb} & \multicolumn{1}{c|}{dadadadawjb} & \multicolumn{1}{c|}{Wang Junbo} \\ \hline
    input & 12345.6 & 5000 & 1020.55 \\ \hline
    output & 765.43 & 5120.5 & 98760 \\ \hline
    net & 11580.17 & -120.5 & -97739.45 \\ \hline
\end{tabular}

The left 
\begin{tabular}{@{}c}
    Hello \\
    World
\end{tabular}
The right

\begin{tabular}{|c|@{\extracolsep{1em}}c|c|c|}
    \hline
    1 & 2 & 3 & 4 \\
    \hline
\end{tabular}

\begin{tabular}[t]{|c|}
    \hline
    up \\ mid \\ down \\
    \hline
\end{tabular}
Ragged top
\begin{tabular}[t]{|c|}     % must usepackage array
    \firsthline
    up \\ mid \\ down \\
    \lasthline
\end{tabular}
Ragged top

\subsubsection{Item Merge And Split}
Used for the horizontal merge:
\begin{tabular}{|r|r|}
    \hline
    \multicolumn{2}{|c|}{grades} \\ \hline
    OS & CA \\ \hline
    95 & 100 \\ \hline
\end{tabular}

Used for changing the ragging:
\begin{tabular}{|r|r|}
    \hline
    \multicolumn{1}{|c|}{input} & \multicolumn{1}{c|}{output} \\ \hline
    1 & 1 \\ \hline
    2 & 4 \\ \hline
    3 & 9 \\ \hline
\end{tabular}

Used for the vertical merge:
\begin{tabular}{|c|r|r|}
    \hline
      & \multicolumn{2}{c|}{grades} \\ \cline{2-3}
    name & OS & CA \\ \hline
    wjb & 95 & 100 \\ \hline
\end{tabular}
\begin{tabular}{|c|r|r|}    % must usepackage multirow
    \hline
    \multirow{2}*{name}  & \multicolumn{2}{c|}{grades} \\ \cline{2-3}
      & OS & CA \\ \hline
    wjb & 95 & 100 \\ \hline
\end{tabular}

Used for the horizontal split:
\begin{tabular}{|c|}
    \hline
    1 \\ \hline
    1 \vline\ 2 \\ \hline
    1 \vline\ 2 \vline\ 3 \\ \hline
\end{tabular}
\begin{tabular}{|c|}
    \hline
    1 \\ \hline
    \begin{tabular}{@{}c|c@{}} 1 & 2 \end{tabular} \\ \hline
    \begin{tabular}{@{}c|c|c@{}} 1 & 2 & 3 \end{tabular} \\ \hline
\end{tabular}

Make item cell:
% must usepackage makecell
\begin{tabular}{|r|r|}
    \hline
    \makecell[c]{input\\data} & \makecell[c]{output\\data} \\ \hline
    12345 & 56789 \\ \hline
\end{tabular}
\begin{tabular}{|r|r|}
    \hline
    \thead[c]{input\\data} & \thead[c]{output\\data} \\ \hline
    12345 & 56789 \\ \hline
\end{tabular}
\settowidth\rotheadsize{\theadfont Mathematica}     % set the width of the head rotated
\begin{tabular}{|c|c|}
    \hline
    \thead[c]{name} & \rothead{math\\grade} \\ \hline
    wjb & 100 \\ \hline
\end{tabular}
\begin{tabular}{|c|r|}
    % must usepackage multirow and makecell
    \hline
    \multirowcell{3}{multiple\\grades} & 100 \\ \cline{2-2}
      & 99 \\ \cline{2-2}
      & 98 \\ \hline
\end{tabular}

Diagnal split:
% must usepackage diagbox
\begin{tabular}{|c|ccc|}
    \hline
    \diagbox{row}{value}{column} & c1 & c2 & c3 \\ \hline
    r1 & 1 & 0 & 0 \\
    r2 & 0 & 1 & 0 \\
    r3 & 0 & 0 & 1 \\ \hline
\end{tabular}

\subsubsection{Width}
\begin{tabular*}{\textwidth}{|c@{\extracolsep{\fill}}ccccc|}
    \hline
    number & 1 & 2 & 3 & 4 & 5 \\
    character & A & B & C & D & E \\
    \hline
\end{tabular*}

\begin{tabularx}{\textwidth}{|c|X|X|X|X|X|}     % must usepackage tabularx
    \hline
    number & 1 & 2 & 3 & 4 & 5 \\ \hline
    character & A & B & C & D & E \\ \hline
\end{tabularx}      % by default ragged right

\begin{tabularx}{\textwidth}{|c|Y|Y|Y|Y|Y|}
    \hline
    number & 1 & 2 & 3 & 4 & 5 \\ \hline
    character & A & B & C & D & E \\ \hline
\end{tabularx}

We can not deal with the too wide tabular.

\subsubsection{Length}
\begin{longtable}[c]{|c|c|}         % must usepackage longtable
    \caption{The Example Of Longtable} \\
    \hline
    \endfirsthead

    \multicolumn{2}{l}{ (Continued) } \\
    \hline
    \endhead

    \multicolumn{2}{c}{Continuing next page\ldots} \\[2ex]
    \endfoot

    \hline
    \multicolumn{2}{r}{End Of Table} \\
    \endlastfoot

    name & description \\ \hline
    a & the fist character in alphabet \\
    b & the second character in alphabet \\
    c & the third character in alphabet \\
    d & the fourth character in alphabet \\
    e & the fifth character in alphabet \\
    f & the sixth character in alphabet \\
    g & the seventh character in alphabet \\
    h & the eighth character in alphabet \\
    i & the ninth character in alphabet \\
    j & the tenth character in alphabet \\
    k & the eleventh character in alphabet \\
    l & the twelfth character in alphabet \\
    m & the thirdth character in alphabet \\
    n & the fourteenth character in alphabet \\
    o & the fifteenth character in alphabet \\
    p & the sixteenth character in alphabet \\
    q & the seventeenth character in alphabet \\
    r & the eighteenth character in alphabet \\
    s & the nineteenth character in alphabet \\
    t & the twentieth character in alphabet \\
    u & the twenty first character in alphabet \\
    v & the twenty second character in alphabet \\
    w & the twenty third character in alphabet \\
    x & the twenty fourth character in alphabet \\
    y & the twenty fifth character in alphabet \\
    z & the twenty sixth character in alphabet \\
\end{longtable}

\LTXtable{\textwidth}{LTXTable.tex}     % must usepackage ltxtable

\begin{VerbatimOut}{LTXTable.vrb}
    \begin{longtable}{|X|X|}
        \caption{The Example Of LTXTable With Fancyvrb} \\
        \hline
        \endfirsthead
    
        \multicolumn{2}{l}{ (Continued) } \\
        \hline
        \endhead
    
        \multicolumn{2}{c}{Continuing next page\ldots} \\[2ex]
        \endfoot
    
        \hline
        \multicolumn{2}{r}{End Of Table} \\
        \endlastfoot
    
        Name & Description \\ \hline
        A & The 1st character in alphabet \\
        B & The 2nd character in alphabet \\
        C & The 3rd character in alphabet \\
        D & The 4th character in alphabet \\
        E & The 5th character in alphabet \\
        F & The 6th character in alphabet \\
        G & The 7th character in alphabet \\
        H & The 8th character in alphabet \\
        I & The 9th character in alphabet \\
        J & The 10th character in alphabet \\
        K & The 11th character in alphabet \\
        L & The 12th character in alphabet \\
        M & The 13th character in alphabet \\
        N & The 14th character in alphabet \\
        O & The 15th character in alphabet \\
        P & The 16th character in alphabet \\
        Q & The 17th character in alphabet \\
        R & The 18th character in alphabet \\
        S & The 19th character in alphabet \\
        T & The 20th character in alphabet \\
        U & The 21st character in alphabet \\
        V & The 22nd character in alphabet \\
        W & The 23rd character in alphabet \\
        X & The 24th character in alphabet \\
        Y & The 25th character in alphabet \\
        Z & The 26th character in alphabet \\
    \end{longtable}
\end{VerbatimOut}
\LTXtable{\textwidth}{LTXTable.vrb}         % must usepackage ltxtable and fancyvrb

\begin{longtabu} to \textwidth {|X|X|}      % must usepackage tabu and longtable
    \caption{The Example Of Longtabu} \\
    \hline
    \endfirsthead

    \multicolumn{2}{l}{ (Continued) } \\
    \hline
    \endhead

    \multicolumn{2}{c}{Continuing next page\ldots} \\[2ex]
    \endfoot

    \hline
    \multicolumn{2}{r}{End Of Table} \\
    \endlastfoot

    Name & Description \\ \hline
    a & The 1st character in alphabet \\
    b & The 2nd character in alphabet \\
    c & The 3rd character in alphabet \\
    d & The 4th character in alphabet \\
    e & The 5th character in alphabet \\
    f & The 6th character in alphabet \\
    g & The 7th character in alphabet \\
    h & The 8th character in alphabet \\
    i & The 9th character in alphabet \\
    j & The 10th character in alphabet \\
    k & The 11th character in alphabet \\
    l & The 12th character in alphabet \\
    m & The 13th character in alphabet \\
    n & The 14th character in alphabet \\
    o & The 15th character in alphabet \\
    p & The 16th character in alphabet \\
    q & The 17th character in alphabet \\
    r & The 18th character in alphabet \\
    s & The 19th character in alphabet \\
    t & The 20th character in alphabet \\
    u & The 21st character in alphabet \\
    v & The 22nd character in alphabet \\
    w & The 23rd character in alphabet \\
    x & The 24th character in alphabet \\
    y & The 25th character in alphabet \\
    z & The 26th character in alphabet \\
\end{longtabu}

\begin{center}
    \tablecaption{The Example Of Xtab}
    \tablefirsthead{\hline}
    \tablehead{\multicolumn{2}{l}{ (Continued) } \\ \hline}
    \tabletail{\multicolumn{2}{c}{Continuing next page\ldots} \\[2ex]}
    \tablelasttail{\hline \multicolumn{2}{r}{End Of Table} \\}
    \begin{xtabular}{|c|c|}     % must usepackage xtab, if `*` then fix width like tabular*
        name & description \\ \hline
        a & The 1st character in alphabet \\
        b & The 2nd character in alphabet \\
        c & The 3rd character in alphabet \\
        d & The 4th character in alphabet \\
        e & The 5th character in alphabet \\
        f & The 6th character in alphabet \\
        g & The 7th character in alphabet \\
        h & The 8th character in alphabet \\
        i & The 9th character in alphabet \\
        j & The 10th character in alphabet \\
        k & The 11th character in alphabet \\
        l & The 12th character in alphabet \\
        m & The 13th character in alphabet \\
        n & The 14th character in alphabet \\
        o & The 15th character in alphabet \\
        p & The 16th character in alphabet \\
        q & The 17th character in alphabet \\
        r & The 18th character in alphabet \\
        s & The 19th character in alphabet \\
        t & The 20th character in alphabet \\
        u & The 21st character in alphabet \\
        v & The 22nd character in alphabet \\
        w & The 23rd character in alphabet \\
        x & The 24th character in alphabet \\
        y & The 25th character in alphabet \\
        z & The 26th character in alphabet \\
    \end{xtabular}
\end{center}

\subsubsection{Thickness}
\begin{tabular}{*{6}{c}}    % must usepackage booktabs
    \toprule
    \multirow{2}*{Name} & \multicolumn{2}{c}{CS} & \multicolumn{2}{c}{SE} &  \\
    \cmidrule(lr){2-3} \cmidrule(lr){4-5} \cmidrule{6-6}\morecmidrules\cmidrule{6-6}
     & OS & CA & SEP & CSE & All \\
    \midrule
    wjb & 100 & 100 & 100 & 100 & A+ \\
    dadadadawjb & 99 & 99 & 99 & 99 & A+ \\
    Wang Junbo & 98 & 98 & 98 & 98 & A+ \\
    \bottomrule
\end{tabular}

\begin{tabular}{c|cc}       % must usepackage makecell
    \Xhline{2pt}
    input & \multicolumn{2}{c}{output} \\
    \Xcline{2-3}{0.4pt}
    x & y & z \\
    \Xhline{1pt}
    1 & 1 & 1 \\
    2 & 4 & 8 \\
    \Xhline{2pt}
\end{tabular}

\begin{tabular}{vc|ccv}     % must usepackage array, bug exists
    \hline
    input & \multicolumn{2}{cv}{output} \\
    \cline{2-3}
    x & y & z \\
    \hline
    1 & 1 & 1 \\
    2 & 4 & 8 \\
    \hline
\end{tabular}

\subsubsection{Double Lines}
Should omit double lines in tabular
\begin{tabular}{|c||cc|}
    \hline\hline
    input & \multicolumn{2}{c|}{output} \\
    \cline{2-3}
    x & y & z \\
    \hline\hline
    1 & 1 & 1 \\
    2 & 4 & 8 \\
    \hline\hline
\end{tabular}
\begin{tabular}{|c||cc|}        % must usepackage hhline
    \hhline{|=:t:==|}
    x & y & z \\
    \hhline{|=::==|}
    1 & 1 & 1 \\
    2 & 4 & 8 \\
    \hhline{|=:b:==|}
\end{tabular}

\subsubsection{Dash Line}
% conflict with hhline and makecell
% \begin{tabular}{:c:cc:}           % must usepackage arydshln and array
%     \firsthdashline
%     input & \multicolumn{2}{c:}{output} \\
%     \cdashline{2-3}
%     x & y & z \\
%     \hdashline
%     1 & 1 & 1 \\
%     2 & 4 & 8 \\
%     \lasthdashline
% \end{tabular}
% \begin{tabular}{;{8pt/2pt}c;{2pt/2pt}cc;{8pt/2pt}}              % must usepackage arydshln and array
%     \firsthdashline
%     input & \multicolumn{2}{c;{8pt/2pt}}{output} \\
%     \cdashline{2-3}[2pt/2pt]
%     x & y & z \\
%     \hdashline[8pt/2pt]
%     1 & 1 & 1 \\
%     2 & 4 & 8 \\
%     \lasthdashline
% \end{tabular}

\subsection{Float Table}
\subsubsection{Overview}
\begin{table}[!htbp]
    \centering
    \bicaption[Tabular In Table]{\label{tab-float} Tabular In Table - Example Of Float Table}[浮动表格]{浮动表格的例子}
    \begin{tabular}{|c|c|c|}
        \hline
        left & center & right \\ \hline
        item1 & item2 & item3 \\ \hline
    \end{tabular}
\end{table}

\subsubsection{Rotated Table}
% must usepackage rotfloat
\begin{sidewaystable}
    \centering
    \begin{tabular}{ccccccccc}
        \hline
        Left & LEFT & left & Middle & MIDDLE & middle & Right & RIGHT & right \\ \hline
        $-4$ & $-3$ & $-2$ & $-1$ & $0$ & $1$ & $2$ & $3$ & $4$ \\ \hline
    \end{tabular}
\end{sidewaystable}

\subsubsection{Side By Side}
\paragraph{Table With Words}
\begin{table}[H]
    \centering
    \caption{Table With Words Example}
    \begin{tabular}{|c|c|c|}
        \hline
        left & center & right \\ \hline
        item1 & item2 & item3 \\ \hline
    \end{tabular}
    \qquad
    \parbox[b]{0.4\textwidth}{This is the comment on the table.}
\end{table}

\paragraph{Tables Side By Side}
\begin{table}[H]
    \centering
    \caption{Table Side By Side Example}
    \begin{tabular}{|c|c|c|}
        \hline
        left & center & right \\ \hline
        item1 & item2 & item3 \\ \hline
    \end{tabular}
    \qquad
    \begin{tabular}{|c|c|c|}
        \hline
        left & center & right \\ \hline
        item1 & item2 & item3 \\ \hline
    \end{tabular}
\end{table}

\paragraph{Captions For Side By Side}
\begin{table}[H]
    \caption{Title For Both}
    \parbox[b]{0.5\textwidth}{\centering
        \caption{Title For Left}
        \begin{tabular}{|c|c|c|}
            \hline
            left & center & right \\ \hline
            item1 & item2 & item3 \\ \hline
        \end{tabular}}
    \parbox[b]{0.5\textwidth}{\centering
        \caption{Title For Right}
        \begin{tabular}{|c|c|c|}
            \hline
            left & center & right \\ \hline
            item1 & item2 & item3 \\ \hline
        \end{tabular}}
\end{table}

\begin{table}[H]
    % must usepackage subcaption
    \caption{Title For Both}
    \begin{subtable}[b]{0.5\textwidth}
        \centering
        \begin{tabular}{|c|c|c|}
            \hline
            left & center & right \\ \hline
            item1 & item2 & item3 \\ \hline
        \end{tabular}
        \caption{Title For Left}
    \end{subtable}
    \begin{subtable}[b]{0.5\textwidth}
        \centering
        \begin{tabular}{|c|c|c|}
            \hline
            left & center & right \\ \hline
            item1 & item2 & item3 \\ \hline
        \end{tabular}
        \caption{Title For Right}
    \end{subtable}
\end{table}

\subsubsection{Table Arrounded By Words}
\begin{tabwindow}[2, r, 
    \begin{tabular}{|c|c|c|}
        \hline
        left & center & right \\ \hline
        item1 & item2 & item3 \\ \hline
    \end{tabular}, Title Of Arounded Table]             % must usepackage picinpar
    % bug exists
    These are the words around the table, and the table will appear below two lines, 
    lying in the right of the words, rather than left. The following will be fillings. 
    Hello world. Hello world. Hello world. Hello world. Hello world. Hello world. Hello world. 
    Hello world. Hello world. Hello world. Hello world. Hello world. Hello world. Hello world. 
    Hello world. Hello world. Hello world. Hello world. Hello world. Hello world. Hello world. 
    Hello world. Hello world. Hello world. Hello world. Hello world. Hello world. Hello world. 
    Hello world. Hello world. Hello world. Hello world. Hello world. Hello world. Hello world. 
    Hello world. Hello world. Hello world. Hello world. Hello world. Hello world. Hello world. 
    Hello world. Hello world. Hello world. Hello world. Hello world. Hello world. Hello world. 
    Hello world. Hello world. Hello world. Hello world. Hello world. Hello world. Hello world. 
    Hello world. Hello world. Hello world. Hello world. Hello world. Hello world. Hello world. 
    Hello world. Hello world. Hello world. Hello world. Hello world. Hello world. Hello world.
\end{tabwindow}

\begin{wraptable}[10]{r}[1.5cm]{5cm}                    % must usepackage wrapfig
    \centering
    \caption{Title Of Arounded Table}
    \begin{tabular}{|c|c|c|}
        \hline
        left & center & right \\ \hline
        item1 & item2 & item3 \\ \hline
    \end{tabular}
\end{wraptable}
These are the words around the table, and the table will appear with height of 10 lines, 
lying in the right of the words, rather than left, and it will extend out 1.5cm, and has width of 5cm. 
The following will be fillings. 
Hello world. Hello world. Hello world. Hello world. Hello world. Hello world. Hello world. 
Hello world. Hello world. Hello world. Hello world. Hello world. Hello world. Hello world. 
Hello world. Hello world. Hello world. Hello world. Hello world. Hello world. Hello world. 
Hello world. Hello world. Hello world. Hello world. Hello world. Hello world. Hello world. 
Hello world. Hello world. Hello world. Hello world. Hello world. Hello world. Hello world. 
Hello world. Hello world. Hello world. Hello world. Hello world. Hello world. Hello world. 
Hello world. Hello world. Hello world. Hello world. Hello world. Hello world. Hello world. 
Hello world. Hello world. Hello world. Hello world. Hello world. Hello world. Hello world. 
Hello world. Hello world. Hello world. Hello world. Hello world. Hello world. Hello world. 
Hello world. Hello world. Hello world. Hello world. Hello world. Hello world. Hello world.

\subsection{Color Table}
% bug exists
% \begin{tabular}{>{\columncolor{gray}}c | >{\columncolor{lightgray}}c}
%     darker & lighter \\
%     gray & gray \\
% \end{tabular}

% \begin{tabular}{c|c}
%     \rowcolor{lightgray} light & gray \\
%     just & white \\
% \end{tabular}

% \begin{tabular}{cccc}
%     No & No & \cellcolor{lightgray}Yes & No \\
%     \cellcolor{lightgray}Yes & No & No & No \\
% \end{tabular}

% \arrayrulecolor{gray}
% \doublerulesepcolor{lightgray}
% \begin{tabular}{cc}
%     \hline\hline
%     gray line & lightgray doubleline \\
%     \arrayrulecolor{black}
%     \doublerulesepcolor{white}
%     \hline
%     original black line & original white double line \\
%     \hline\hline
% \end{tabular}
% \arrayrulecolor{black}                              % restore
% \doublerulesepcolor{white}

% \rowcolors[\hline]{2}{black!20}{black!10}
% \begin{tabular}{crr}
%     \rowcolor{black!30} person & OS & CA \\         % set the color of table header separately
%     dadadadawjb & 100 & 100 \\
%     Junbo Wang & 99 & 99 \\
%     wjb & 98 & 98 \\
% \end{tabular}