%%%%%%%%%%---------- Fonts ----------%%%%%%%%%%
% \usepackage{amsthm}                     % for using some theorem environment, must before fontspec, incompatible with ntheorem

% \frenchspacing                          % for forbidding extra blank after puctuations, only the same blank length between the normal words
% \CJKsetecglue{}                         % for forbidding auto-blank between chinese and other contents completely
% \renewcommand\nocorrlist{,.A}           % for not correcting the specific characters' slanted overflow

\usepackage{fontspec}                   % for using xetex setting font family, if `no-math`, then no influencing the math font
\setmainfont{Times New Roman}
\setsansfont{Verdana}
\setmonofont{Courier New}
\newfontfamily\lucidasans{Lucida Sans}  % for define new font family, used by \lucidasans
\setmathrm{Latin Modern Math}
\setmathsf{Verdana}
\setmathtt{Courier New}

\newcommand\Emph{\textbf}               % for defining more emphasize, used by \Emph
\usepackage[normalem]{ulem}             % for correct underline, if `normalem`, then no influencing the \emph, used by \uline
\usepackage{xeCJKfntef}                 % for using underline chinese, if `normalem`, then no influencing the \emph

\setCJKmainfont{SimSun}
\setCJKsansfont{YouYuan}
\setCJKmonofont{FangSong}
\setCJKfamilyfont{kai}{KaiTi}           % for defining new chinese font family, used by \CJKfamily{kai}

\linespread{1}                          % set line spacing
\selectfont

%%%%%%%%%%---------- Blank ----------%%%%%%%%%%
\newlength\mylen                        % define own length variable
\setlength\mylen{2em}

%%%%%%%%%%---------- Box ----------%%%%%%%%%%
\newsavebox\mybox                       % define own box variable, used by \mybox
\sbox\mybox{My box contents}
\savebox\mybox[5cm][s]{My box contents} % c for center, l for left, r for right, s for stretched
% \usepackage{varwidth}
% \usepackage{graphicx}

%%%%%%%%%%---------- Paragraph ----------%%%%%%%%%%
\usepackage{indentfirst}                % for setting paragraph indent
\setlength{\parindent}{2em}

\hyphenation{man-u-script com-pu-ter}   % manuscript set the hyphenation

\usepackage{ragged2e}                   % for paragraph setting's supplement

\usepackage{microtype}                  % for better english uneven short lines, just included, no need to use explicitly

\usepackage{lettrine}                   % for first word sanking, used by \lettrine

\usepackage{shapepar}                   % for editing the paragraph shape

%%%%%%%%%%---------- Word Environment ----------%%%%%%%%%%
\renewcommand\abstractname{Name for abstract}   % for renaming the title of abstract
\ctexset{abstractname={Name for abstract}}      % for renaming the title of abstract in Chinese

%%%%%%%%%%---------- List ----------%%%%%%%%%%
\renewcommand\theenumii{\fnsymbol{enumii}}      % for changing the index of enumerate environment
\renewcommand\labelenumii{(\theenumii)}         % for changing the tag of enumerate environment, similar for itemize environment

\newenvironment{myitemize}{%
    \begin{list}{\textbullet}{%
        \setlength\topsep{0pt} \setlength\partopsep{0pt} \setlength\parsep{0pt} \setlength\itemsep{0pt}}}
    {\end{list}}                                % general list difficult to use, often used for new own environment

\usepackage{enumitem}                           % for editting easily enumerate environment
\AddEnumerateCounter{\chinese}{\chinese}{}      % for processing no-standard counter

%%%%%%%%%%---------- Counter ----------%%%%%%%%%%
\newcounter{mycnt}[section]                     % for defining own counter variable, if added another counter, then that counter step will cause mycnt clear
\setcounter{mycnt}{0}
\renewcommand\themycnt{\thesection-\arabic{mycnt}}
% used by \stepcounter{mycnt}\themycnt, \addtocounter{mycnt}{-1}\themycnt, often as part of own defined command

\usepackage[centertags, sumlimits, nointlimits, namelimits]{amsmath}
% centertags or tbtags, sumlimits or nosumlimits, intlimits or nointlimits, namelimits or nonamelimits
\numberwithin{equation}{section}                % for making equation indexing along with section

\usepackage{chngcntr}                           % similar to amsmath
\counterwithin{figure}{section}                 % for making figure indexing along with section
\counterwithout{figure}{section}                % for making figure indexing not along with section

%%%%%%%%%%---------- Theorem ----------%%%%%%%%%%
% \usepackage{theorem}                            % for changing the style of theorem environment, must above newtheorem
\usepackage[thmmarks]{ntheorem}                 % better than theorem, added theoremstyle nonumberplain, nonumberbreak, empty, if `thmmarks`, then can use \theoremsymbol
\theoremstyle{plain}                            % also plain, break, marginbreak, changebreak, change, margin
\theoremheaderfont{\sffamily\bfseries}
\theorembodyfont{\normalfont}
\setlength\theorempreskipamount{5pt}
\setlength\theorempostskipamount{5pt}
\theoremsymbol{\ensuremath{\boxed{\phantom{n}}}}

% \usepackage{amsthm}
% \renewcommand\proofname{ProofTitle}             % used by \begin{proof}
% \renewcommand\qedsymbol{\ensuremath{\boxed{\phantom{M}}}}

\newtheorem{thm}{TheoremTitle}[section]         % for creating a new theorem environment, used by \begin{thm}, if added another counter, then that counter step will cause thm clear
% if the counter is located in the middle, then the index is determined by the counter, often used for different theorem with same index

%%%%%%%%%%---------- Verbatim ----------%%%%%%%%%%
\newsavebox\verbbox                             % used for verbatim
\begin{lrbox}\verbbox
    \verb!#$%^&{}!
\end{lrbox}

\newsavebox\verbatimbox                         % used for verbatim multiple lines
\begin{lrbox}\verbatimbox
    \begin{minipage}{10em}
        \begin{verbatim}
            #$%^&{}
            #$%^&{}
            #$%^&{}
        \end{verbatim}
    \end{minipage}
\end{lrbox}

\usepackage{fancyvrb}                           % used for verbatim
\SaveVerb{myverb}!#^&{}!                        % used by \UseVerb

\usepackage{cprotect}                           % for protecting verbatim as other commands' parameter

\usepackage{verbatim}                           % for verbatiming whole file

\usepackage{shortvrb}                           % for short writing the verbatim
% \MakeShortVerb!                                 % used by directly !something!, danger to some degree

%%%%%%%%%%---------- Codes ----------%%%%%%%%%%
\usepackage{listings}                           % for codes environment, used by \begin{lstlisting}
\lstset{
    basicstyle=\sffamily,
    keywordstyle=\bfseries,
    commentstyle=\rmfamily\itshape,
    stringstyle=\ttfamily,
    flexiblecolumns,                            % same as columns=flexible
    numbers=left, numberstyle=\footnotesize,
    escapechar=`,
    language=C
}

%%%%%%%%%%---------- Tabular ----------%%%%%%%%%%
\usepackage{dcolumn}                            % for using tabular aligned with math dot, will also usepackage array by default
\newcolumntype{d}{D{.}{.}{2}}                   % for changing the use of math dot, input as ., output as ., allow 2 bits

\setlength\tabcolsep{1em}                       % for changing the length between columns in tabular
\setlength\arraycolsep{1em}                     % for changing the length between columns in array
\renewcommand\arraystretch{2}                   % for changing the length between rows in tabular and array, by default 1

% \usepackage{array}                              % already include by default with dcolumn

\usepackage{multirow}                           % for using multirow tabular

\usepackage{rotating, makecell}                 % for using tabular item cell

\usepackage{diagbox}                            % for using diagnal tabular item

\usepackage{tabularx}                           % for using fixed width tabular, used by \begin{tabularx}
\newcolumntype{Y}{>{\centering\arraybackslash}X}    % for changing the style of tabularx

\usepackage{longtable}                          % for using long tabular, used by \begin{longtable}

\usepackage{ltxtable}                           % for using long tabular with tabularx, used by \LTXtable

\usepackage[delarray]{tabu}                     % for using tabular more easily, must also together with longtable

\usepackage{xtab}                               % for using long tabular also, used by \begin{xtabular} or \begin{xtabular*}

\usepackage{booktabs}                           % for using different thickness tabular line, almost for three-line table
\setlength\heavyrulewidth{0.08em}               % for changing the thickness of \toprule and \bottomrule
\setlength\lightrulewidth{0.05em}               % for changing the thickness of \midrule
\setlength\cmidrulewidth{0.03em}                % for changing the thickness of \cmidrule
\setlength\aboverulesep{0.03em}                 % for changing the length above \bottomrule and \midrule and \cmidrule
\setlength\belowrulesep{0.03em}                 % for changing the length below \toprule and \midrule and \cmidrule
\setlength\abovetopsep{0em}                     % for changing the length above \toprule
\setlength\belowbottomsep{0em}                  % for changing the length below \bottomrule

\setlength\arrayrulewidth{1pt}                  % for changing the whole lines thichness in tabular

\newcolumntype{v}{!{\vrule width 2pt}}          % for using different thickness vertical lines in tabular

\setlength\doublerulesep{1pt}                   % for changing the length between double lines in tabular

\usepackage{hhline}                             % for using correct double lines in tabular

% \usepackage{arydshln}                           % for using dash line in tabular, conflict with hhline and makecell
% \newcolumntype{|}{!{\vline}}                    % for solving the conflict between arydshln and hhline, makecell, bug exists

% \setlength\dashlinedash{4pt}                    % for changing the length of dash in dash line in tabular
% \setlength\dashlinegap{4pt}                     % for changing the length of gap in dash line in tabular

\usepackage{rotfloat}                           % for using rotated table and figure, used by \begin{sidewaystable} and \begin{sidewaysfigure}

\usepackage{caption, subcaption}                % for changing the style of the caption of float
\captionsetup[table]{format=hang, labelformat=brace, labelsep=endash, justification=centering, singlelinecheck=false, 
                    font={small,sf}, labelfont=sc, textfont=bf, width=20em, skip=10pt, name=TableName}
\usepackage{bicaption}                          % for using bilingual caption
\DeclareCaptionOption{english}[]{%
    \renewcommand\figurename{FigureName}
    \renewcommand\tablename{TableName}}
\DeclareCaptionOption{chinese}[]{%
    \renewcommand\figurename{图名}
    \renewcommand\tablename{表名}}
\captionsetup[bi-first]{english}
\captionsetup[bi-second]{chinese}
\captionsetup[sub]{font={small, it}}                            % set for all subcaptions
\captionsetup[subtable]{labelformat=simple, labelsep=colon}     % set for subtables

\usepackage{picinpar}                           % for using table and figure arounded by words, used by \begin{tabwindow} and \begin{figwindow}
\usepackage{wrapfig}                            % for using table and figure arounded by words, used by \begin{wraptable} and \begin{wrapfigure}

% \usepackage{colortbl}                           % for using color in table, must after xcolor

%%%%%%%%%%---------- Notes ----------%%%%%%%%%%
\renewcommand\thefootnote{\fnsymbol{footnote}}                      % for changing the form of the index of footnote
\renewcommand\thempfootnote{\textcircled{\arabic{mpfootnote}}}      % for changing the form of the index of minipage footnote

\usepackage{pifont}                                                 % for easier using circled index
\renewcommand\thempfootnote{\ding{\numexpr171+\value{mpfootnote}}}  % for changing the form of the index of minipage footnote into circled number

\setlength\footnotesep{6.65pt}                  % for editting the length between footnotes
\renewcommand\footnoterule{                     % for editting the format of the footnote line
    \vspace*{0.3cm}
    \hrule width 2.5cm
    \vspace*{0.3cm}
}

\usepackage[perpage]{footmisc}                  % for footnote indexing every page, while every chapter by default

\setlength\marginparwidth{6em}                  % for changing the width of marginnote
\setlength\marginparsep{1em}                    % for changing the length between marginnote and article
\setlength\marginparpush{1em}                   % for changing the length between marginnotes

\setlength\smallskipamount{3pt plus 1pt minus 1pt}  % for changing the length of \smallskip
\setlength\medskipamount{2\smallskipamount}         % for changing the length of \medskip
\setlength\bigskipamount{4\smallskipamount}         % for changing the length of \bigskip

\raggedbottom                                   % for setting newpage no need for alignment
\flushbottom                                    % for setting newpage should alignment

\usepackage{varwidth}                           % for using natrual minipage, used by \begin{varwidth}

%%%%%%%%%%---------- LaTeX Logos ----------%%%%%%%%%%
\usepackage{mflogo}                             % for using logos, \MF and \MP
\usepackage{metalogo}                           % for using logos, \XeTeX and \XeLaTeX
% \usepackage{doc}                                % for using logos, \BibTeX, bug exists

%%%%%%%%%%---------- Structure Demostration ----------%%%%%%%%%%
\ctexset{today=old}                             % for changing the format of \today, can be small, big, old, by default small

\usepackage{authoraftertitle}                   % for using title, author, date in document, used by \MyTitle, \MyAuthor, \MyDate

\setcounter{secnumdepth}{3}                     % for changing the structure levels that can be indexed, by default 3
\setcounter{tocdepth}{3}                        % for changing the structure levels that can be included into table of contents, by default 3

% \usepackage{syntonly}                           % for only doing syntax check
% \syntaxonly

\usepackage[sf, bf, it, raggedright]{titlesec}                          % for changing the format of title
% \titlelabel{\S~\thetitle\quad}                                          % for changing the format of label of title
\titleformat*{\section}{\large\sffamily\bfseries\itshape\centering}     % for changing the format of specific title

% for changing the format of title in Chinese
\ctexset{section = 
            {name={Sec\quad, .}, number={\arabic{section}}, format={\centering\bfseries\Large}, 
            numberformat+={\rmfamily}, titleformat+={\rmfamily}, nameformat+={\rmfamily}, 
            aftername={\hspace{1ex}---\hspace{1ex}}, beforeskip={2em}, afterskip={2em}}}
\ctexset{subsection = 
            {name={\fnsymbol{subsection}, :}, format={\raggedright\bfseries\large}, 
            numberformat+={\sffamily}, titleformat+={\sffamily}, nameformat+={\sffamily}, 
            aftername={\hspace{2ex}}, beforeskip={1em}, afterskip={1em}}}
\ctexset{subsubsection = 
            {numberformat+={\ttfamily}, titleformat+={\ttfamily}, nameformat+={\ttfamily}, 
            aftername={\hspace{1ex}}, beforeskip={}, afterskip={}}}

\usepackage{geometry}                           % for changing the size of page
\geometry{a4paper, left=3cm, right=3cm}

\pagenumbering{arabic}                          % for changing the style of page number

\usepackage{fancyhdr}
\pagestyle{fancy}                               % for changing the style of page heading, including empty, plain, headings, myheadings, fancy while fancy need usepackage fancyhdr
\fancyhf{}                                      % clear all page headings and footings
\fancyhead[L]{\slshape \MyTitle}
\fancyhead[R]{\slshape \MyDate}
\fancyhead[C]{\slshape \bfseries\leftmark}
\fancyfoot[C]{page \arabic{page} of \pageref{LastPage}}
\fancyfoot[LR]{$\heartsuit$}
% also can use \lhead, \chead, \rhead, \lfoot, \cfoot, \rfoot
\renewcommand\headrulewidth{0.4pt}              % for changing the fancy heading horizontal line
\renewcommand\footrulewidth{0pt}                % for changing the fancy footing horizontal line
\fancypagestyle{plain}{%
    \fancyhf{}
    \cfoot{---\textit{\thepage}---}
    \renewcommand\headrulewidth{0pt}
    \renewcommand\footrulewidth{0pt}
}

\setlength\headheight{13pt}                     % for fixing fancyhdr's warning

\setlength\columnsep{3em}                       % for changing the length between two columns
\setlength\columnseprule{0.4pt}                 % for changing the width of column vertical line, by default 0pt

% \usepackage{balance}                            % for changing the mode of non-balance two columns mode
% \balance                                        % otherwise use \nobalance to return non-balance two columns mode

\usepackage{multicol}                           % for using more powerful columns mode, conflicting with floating and marginpar, used by \begin{multicols}

\usepackage{pdflscape}                          % for using rotated page, used by \begin{landscape}

%%%%%%%%%%---------- Macros ----------%%%%%%%%%%
\newcommand\wjb{\emph{Wang Junbo}}
\newcommand\loves[2]{#1 loves #2.}
\newcommand\vloves[3][loves]{#2 #1 #3.}

\renewcommand\wjb{\emph{Junbo Wang}}

\newenvironment{myquotation}[1]{%
    \newcommand\quotesource{#1}
    \begin{quotation}
}{%
    \par\hfill ------ 《\textit{\quotesource}》
    \end{quotation}
}

%%%%%%%%%%---------- Table Of Contents ----------%%%%%%%%%%
% \usepackage{tocbibind}                          % for adding more to table of contents, then can not include \addcontentsline

\usepackage{tocloft}                            % for changing the style of table of contents
\tocloftpagestyle{fancy}                        % for changing the style of the page of table of contents, by default plain

\renewcommand\contentsname{Title of Table of Contents}      % for changing the format of the title of table of contents
\setlength\cftbeforetoctitleskip{2ex}
\setlength\cftaftertoctitleskip{2ex}
\renewcommand\cfttoctitlefont{\hfill\Large\sffamily\bfseries}
\renewcommand\cftaftertoctitle{\hfill}

\renewcommand\cftdot{$\ast $}                   % for changing the format of the items of table of contents, leaving several options not listed
\renewcommand\cftdotsep{1}

%%%%%%%%%%---------- Cross Reference ----------%%%%%%%%%%
\usepackage{hyperref}
\hypersetup{%
    colorlinks=true,
    bookmarks=true,
    bookmarksopen=true,
    bookmarksnumbered=true,
    pdfborder={0 0 1},
    pdfpagemode=FullScreen,
    pdfstartview=Fit,
    pdftitle={Latex How To},
    pdfauthor={Junbo Wang},
    pdfsubject={How to write Latex},
    pdfkeywords={HowTo, Latex}
}

\usepackage{nameref}                            % for using reference name, used by \nameref

\usepackage{lastpage}                           % for using the total number of pages, used by \pageref{LastPage}

\usepackage{xr}                                 % for using cross reference from external .aux file
\externaldocument[ex:]{ExternalCrossReference}  % if option, then add this as the prefix when reference

%%%%%%%%%%---------- Literature Reference ----------%%%%%%%%%%
% \bibliographystyle{alpha}                       % for changing the style of literature reference, also can use plain, unsrt, alpha, abbrv

\usepackage[numbers, square, sort&compress]{natbib}             % for using more bib styles
\bibliographystyle{plainnat}                    % for changing the style of literature reference, also can use plainnat, unsrtnat, abbrvnat
\setcitestyle{numbers, square, comma, aysep={;}, yysep={,}, notesep={,}}    % for changing the style of citing
\ctexset{bibname={Title of Literature References}}              % for changing the title of literature references list, by default References
% \renewcommand\bibname{Title of Literature References}           % the same as \ctexset{bibname={Title of Literature References}}
\renewcommand\bibsection{\section*{\refname}}                   % for changing the way to typeset the literature references list, by default the same
\renewcommand\bibpreamble{Literatures referenced as follows.}   % for adding some additional description, by default none
\renewcommand\bibfont{\small}                                   % for changing the font of the literature references list, by default none
\renewcommand\citenumfont{\itshape}                             % for changing the font of the citing numbers, by default none
\renewcommand\bibnumfmt[1]{[\textbf{#1.}]}                      % for changing the style of the index of the literature references list, by default [#1]
\setlength\bibhang{2em}                                         % for changing the hang indent of the literature references list
\setlength\bibsep{0.5em}                                        % for changing the length between items of the literature references list

%%%%%%%%%%---------- Index ----------%%%%%%%%%%
% \usepackage{makeidx}                          % for using index list
\usepackage{imakeidx}                           % for using more powerful index list, used by \index
% \usepackage{idxlayout}                        % for changing the style of index list, must after imakeidx, here we do not use
\makeindex[%
    name=index1,                                % by default \jobname
    title={Title of Index List 1},              % by default \indexname
    intoc=true,                                 % by default false
    columns=1,                                  % by default 1
    program=makeindex,                          % can be also makeindex, xindy, texindy
    options={-s mkind.ist},                     % by default none
    noautomatic=false                           % by default false
]
% more than 1 index lists must usepackage imakeidx
\makeindex[%
    name=index2,
    title={Title of Index List 2},
    intoc=true,
    columns=2,
    columnsep=35pt,                             % by default 35pt
    columnseprule=true,                         % by default false
    program=makeindex,
    options={-s mkind.ist},
    noautomatic=false
]
\indexsetup{%
    level=\section*,                                    % by default \chapter*
    toclevel=section,                                   % by default chapter
    firstpagestyle=plain,                               % by default plain
    headers={Index List odd}{Index List even},          % odd and even page heading
    othercode={ \renewcommand\indexspace{\smallskip} }  % the code excuted before index item
}

\renewcommand\seename{see}                      % for changing the |see output, by default the same
\renewcommand\alsoname{see also}                % for changing the |seealso output, by default the same

\newcommand*\mysee[2]{#2 (\emph{see} #1)}       % for defining a new see, used by |mysee{other_index}

% \usepackage{glossaries}                         % for using glossaries functions, used by \newglossaryentry
% \makeglossaries                                 % similar to \makeindex

%%%%%%%%%%---------- Math ----------%%%%%%%%%%
% \usepackage{amsmath}                            % already usepackage above
\usepackage{mathtools}                          % for using other math tools
\usepackage{tensor}                             % for using special superscript and subscript, used by \indices and \tensor
\usepackage[version=4]{mhchem}                  % for using chemistry formula, used by \ce
\usepackage{xfrac}                              % for using small inline fraction, used by \sfrac

\newcommand\stiring[2]{\genfrac{[}{]}{0pt}{}{#1}{#2}}       % for defining general fraction
\newcommand\dstiring[2]{\genfrac{[}{]}{0pt}{0}{#1}{#2}}     % here for stiring number
\newcommand\tstiring[2]{\genfrac{[}{]}{0pt}{1}{#1}{#2}}     % must usepackage amsmath

\setcounter{MaxMatrixCols}{15}                  % for changing the max columns in matrix, by default 10

\usepackage{delarray}                           % for using delimit directly in matrix

\usepackage{blkarray}                           % for using matrix more easily
\makeatletter
\newbox\BA@first@box
\makeatother                                    % for dealing with the conflict with amsmath

\usepackage{amssymb}                            % for using more math fonts

\newcommand\mi{\mathrm{i}}                      % for imaginary unit `i` in math mode
\newcommand\me{\mathrm{e}}                      % for math constant `e` in math mode
\newcommand\diff{\,\mathrm{d}}                  % for differential operator `d` in math mode

\usepackage{upgreek}                            % for using upright greek letters

\usepackage{amsxtra}                            % for using command \sp

\usepackage{yhmath}                             % for using more math accents

\usepackage{accents}                            % for using more general math accents, used by \accentset and \underaccent

\usepackage{bm}                                 % for using bold math, used by \bm and \hm

\DeclareMathOperator{\card}{card}               % for customizing math operators, must usepackage amsmath, if `*` then with limit
\DeclareMathOperator{\dif}{d\!}                 % for better differential operator `d` in math mode

\newcommand\defeq{\stackrel{\text{d}}{=}}       % for using stack relation

\usepackage{extarrows}                          % for using stackable arrows

\usepackage{mathdots}                           % for using math dots

\setlength\multlinegap{3em}                     % for changing the length between the first line of multiline math environment with the edge
\setlength\multlinetaggap{3em}                  % for changing the length between the last line of multiline math environment with the edge

% \usepackage[mathstyleoff]{breqn}                % for using automatic math line break, bug exists

\usepackage{cases}                              % for using numcases, used by \begin{numcases}

\newtagform{bracket}[\textit]{[}{]}             % for changing the style of the index of math equation
\usetagform{bracket}                            % must usepackage mathtools

\renewcommand\theequation{\thesection.\arabic{equation}}    % for changing the style of the counter of math equation

\renewcommand\*{\discretionary{\,\mbox{$\cdot$}}{}{}}       % for changing thestyle of the math hyphenation \*

\allowdisplaybreaks[1]                          % for setting whether allow multiple line math formula to page break

% \setlength\mathindent{1pt}                      % for changing the indent before math formula when choose fleqn
\setlength\abovedisplayskip{12pt plus 3pt minus 9pt}
\setlength\abovedisplayshortskip{0pt plus 3pt}              % for changing the length between math formula and above text
\setlength\belowdisplayskip{12pt plus 3pt minus 9pt}
\setlength\belowdisplayshortskip{7pt plus 3pt minus 4pt}    % for changing the length between math formula and below text
\setlength\jot{3pt}                             % for changing the length between multiple lines math formula
% \setlength\thinmuskip{3mu}                      % for changing the thin math skip, bug exists
% \setlength\medmuskip{4mu plus 2mu minus 4mu}    % for changing the medium math skip, bug exists
% \setlength\thickmuskip{5mu plus 5mu}            % for changing the thick math skip, bug exists
\setlength\mathsurround{0pt}                    % for changing the length between inline math formula and surrounded text

%%%%%%%%%%---------- Figure ----------%%%%%%%%%%
\usepackage{graphicx}
\graphicspath{{Figures/pdf}{Figures/png}{Figures/jpg}}      % set the path where the figures to be searched

\newfloat{myfloat}{htbp}{lomyfloat}[section]        % create a new float named myfloat, with aux file *.lomy, counter with section, used by \begin{myfloat}
\floatname{myfloat}{Float Title}                    % set the title for myfloat
\floatstyle{ruled}                                  % for changing the style of created float, can be `plain`, `plaintop`, `boxed`, `ruled`
\restylefloat{myfloat}                              % floatstyle only makes sense to the floats defined after it, use restylefloat to set pre-defined float
\floatplacement{myfloat}{H}                         % for resetting default placement

\usepackage{newfloat}                               % for using more advanced float environment
\DeclareFloatingEnvironment[name=mynewfloat, listname=List of mynewfloat, fileext=lomynewfloat, placement=H, within=section]{mynewfloat}    % create a new float named mynewfloat, used by \begin{mynewfloat}

\usepackage[section]{placeins}                      % for setting float barrier, used by \FloatBarrier, `section` for automatically add barrier before each section

\usepackage{afterpage}                              % for setting contents exactly at the top of next page

% \usepackage{endfloat}                               % for making all floats appearing in the end

% \usepackage[table]{xcolor}                          % for using colors, `table` for colortbl, bug exists
\usepackage{xcolor}
\colorlet{darkred}{red!50!black}                    % for defining color

% \usepackage{colortbl}                               % for using color in table, must after xcolor

\usepackage[all, pdf]{xy}                           % for using xy-pic

\usepackage{pstricks}                               % for using pstricks
\usepackage{pstricks-add}

\usepackage{tikz}                                   % for using tikz