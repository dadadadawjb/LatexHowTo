\section{Math}
\subsection{Overview}
\subsubsection{Inline Math Formula}
$a+b=b+a$, \(a+b=b+a\), \begin{math}a+b=b+a\end{math}.
% punctuation inline outside the math mode

\subsubsection{Displayed Math Formula}
$$
a+b=b+a,
$$
\[
    a+b=b+a,
\]
\begin{displaymath}
    a+b=b+a.
\end{displaymath}
% punctuation displayed inside the math mode

\subsubsection{Ensure Math Mode}
% In plain text: \Box; In math mode: \ensuremath{\Box}, also $\ensuremath{\Box}$.   % bug exists

\subsubsection{Indexed Math Formula}
Indexed math formula:
\begin{equation}
    a+b=b+a \label{eq:commutative}  % label is not necessary
\end{equation}

None indexed math formula:
\begin{equation*}
    a+b=b+a % must usepackage amsmath, the same as \[\]
\end{equation*}

\subsubsection{Text In Math Mode}
$ a + b = b + a \text{, where } a = 1 \text{ and } b = 2 $.

\subsection{Structure}
\subsubsection{Superscript And Subscript}
$ a_i = i^2 $

$ A_{ij} = 2^{i+j} $

$ A_i^k = B^k_i $

$ K_{n_i} = K_{2^i} = 2^{n_i} = 2^{2^i} $

\paragraph{Prime}
$ a' $, $ a'' $, $ a_0' = a'_0 $, $ {a'}^2 \neq  a'^2 $

\paragraph{Digree}
$ 90^\circ $

\paragraph{Position}
$ \max_n f(n) = \sum_{i=0}^n a_i = \int_0^n A(t)dt $, $ \max\limits_n f(n) = \sum\limits_{i=0}^n a_i = \int\limits_0^n A(t)dt $

\[
    \max_n f(n) = \sum_{i=0}^n a_i = \int_0^n A(t)dt
\]
\[
    \max\nolimits_n f(n) = \sum\nolimits_{i=0}^n a_i = \int\limits_0^n A(t)dt
\]

$ \prescript{n}{m}{H}_i^j $     % must usepackage mathtools

\[
    \sideset{_a^b}{_c^d} \sum_{i=1}^{n} A_i = \sideset{}{'}{\prod_k} f_i        % must usepackage amsmath
\]

$ \overset{*}{X} $, $ \underset{*}{X} $, $ \overset{*}{\underset{\dag}{X}} $    % must usepackage amsmath

$ A_m^n \neq A_m{}^n $

$ M\indices{^a_b^{cd}_e} $, $ \tensor[^a_b^c_d]{M}{^a_b^c_d} $  % must usepackage tensor

\paragraph{Mhchem}
\ce{H2O}, \ce{CH3COO-}, \ce{^{227}_{90}Th}.     % must usepackage mhchem

\begin{equation}
    \ce{2H2 + O2 ->[\text{fire}] 2H2O}
\end{equation}

\subsubsection{Overline And Underline}
$ \overline{a + b} = \overline{a} + \overline{b} $

$ \underline{a - b} = \underline{a} - \underline{b} $

$ \overbrace{a + b + \cdots + c}^{n+1} = \underbrace{0 + 0 + \cdots + 0}_{n} + 1 $

$ \overbracket{a + b + \cdots + c}^{n+1} = \underbracket{0 + 0 + \cdots + 0}_{n} + 1 $      % must usepackage mathtools

$ \overleftarrow{a + b} $, $ \overrightarrow{a + b} $, $ \overleftrightarrow{a + b} $       % must usepackage amsmath

$ \underleftarrow{a - b} $, $ \underrightarrow{a - b} $, $ \underleftrightarrow{a - b} $

\subsubsection{Fraction}
Inline size: $ \frac{a}{b} $.

Displayed size:
\[
    \frac{a}{b}
\]

Inside size:
\[
    \frac{1}{\frac{1}{2}(a + b)} = \frac{2}{a + b}
\]

Change size:
\[
    \tfrac{a}{b} = \frac{1}{\dfrac{b}{a}}   % must usepackage amsmath
\]

Continued fraction:
\[
    \cfrac{1}{1 + \cfrac{2}{1 + \cfrac{3}{1 + x}}} = \cfrac[c]{1}{1 + \cfrac[l]{2}{1 + \cfrac[r]{3}{1 + x}}}    % must usepackage amsmath
\]

Inline ambiguity: $ 1 / a + b $ and $ \sfrac{1}{a} + b $        % must usepackage xfrac

\paragraph{Binomial}
% must usepackage amsmath
Inline size: $ \binom{a}{b} $.

Displayed size:
\[
    \binom{a}{b}
\]

Inside size:
\[
    \frac{1}{\binom{a}{b}}
\]

Change size:
\[
    \tbinom{a}{b} \neq \frac{1}{\dbinom{b}{a}}
\]

\paragraph{General Fraction}
\[
    \stiring{a}{b} = \dstiring{a}{b} = \tstiring{a}{b}
\]

\subsubsection{Root}
$ \sqrt{4} = \sqrt[3]{8} $

\[
    \sqrt[n]{\frac{x}{y}}
\]
\[
    \sqrt[\uproot{16}\leftroot{-2}n]{\frac{x}{y}}   % must usepackage amsmath
\]

$ \sqrt{\frac{1}{2}} < \sqrt{ \vphantom{\frac{1}{2}} 2} = \sqrt{2} $

$ \sqrt{b} \sqrt{y} $, $ \sqrt{\mathstrut b} \sqrt{\mathstrut y} $

\subsubsection{Matrix}
% must usepackage amsmath
\[
    A = \begin{matrix}
            a_{11} & a_{12} & a_{13} \\
            a_{21} & a_{22} & a_{23} \\
            a_{31} & a_{32} & a_{33}
        \end{matrix} \qquad
    B = \begin{bmatrix}
            b_{11} & b_{12} & b_{13} \\
            b_{21} & b_{22} & b_{23} \\
            b_{31} & b_{32} & b_{33}
        \end{bmatrix} \qquad
    C = \begin{vmatrix}
            c_{11} & c_{12} & c_{13} \\
            c_{21} & c_{22} & c_{23} \\
            c_{31} & c_{32} & c_{33}
        \end{vmatrix}
\]
\[
    D = \begin{pmatrix}
            d_{11} & d_{12} & d_{13} \\
            d_{21} & d_{22} & d_{23} \\
            d_{31} & d_{32} & d_{33}
        \end{pmatrix} \qquad
    E = \begin{Bmatrix}
            e_{11} & e_{12} & e_{13} \\
            e_{21} & e_{22} & e_{23} \\
            e_{31} & e_{32} & e_{33}
        \end{Bmatrix} \qquad
    F = \begin{Vmatrix}
            f_{11} & f_{12} & f_{13} \\
            f_{21} & f_{22} & f_{23} \\
            f_{31} & f_{32} & f_{33}
        \end{Vmatrix}
\]
% no need usepackage
\[
    \bordermatrix{
          & 1 & 2 & 3 \cr
        1 & A & B & C \cr
        2 & D & E & F \cr
    }
\]

\[
    \begin{pmatrix}
        \begin{matrix} 1 & 0 \\ 0 & 1 \end{matrix} & \text{\Large 0} \\
        \text{\Large 0} & \begin{matrix} 1 & 0 \\ 0 & 1 \end{matrix}
    \end{pmatrix}
\]

$ \left(
    \begin{smallmatrix}
        x & -y \\ y & x
    \end{smallmatrix}
\right) $

\paragraph{Dots In Matrix}
\[
    \begin{bmatrix}
        a_{11} & \dots  & a_{1n} \\
               & \ddots & \vdots \\
        \hdotsfor{3}             \\
        0      &        & a_{nn}
    \end{bmatrix}
\]

\paragraph{Matrix In Superscript And Subscript}
\[
    \sum_{\substack{0<i<n \\ 0<j<i}} A_{ij}
\]
\[
    \sum_{\begin{subarray}{l} i<10 \\ j<100 \\ k<1000 \end{subarray}} X(i, j, k) \qquad
    \sum_{\begin{subarray}{c} i<10 \\ j<100 \\ k<1000 \end{subarray}} Y(i, j, k) \qquad
    \sum_{\begin{subarray}{r} i<10 \\ j<100 \\ k<1000 \end{subarray}} Z(i, j, k)
\]

\paragraph{Alignment In Matrix}
% must usepackage mathtools
\[
    \begin{pmatrix*}[r]
        10 & -10 \\ -20 & 3
    \end{pmatrix*}
\]

\paragraph{Array For Matrix}
\[
    \begin{array}({cc}]     % must usepackage delarray
        1 & 2 \\
        3 & 4
    \end{array}
\]

% \[
%     \begin{tabu}[{cc})      % must usepackage tabu with delarray option, bug exists
%         1 & 2 \\
%         3 & 4
%     \end{tabu}
% \]

\[
    \begin{blockarray}{(cc]}    % must usepackage blkarray
        1 & 2 \\
        3 & 4
    \end{blockarray}
\]

\[ \left[
    \begin{blockarray}{*4r}     % must usepackage blkarray
        \begin{block}{(rr)rr}
            a & -b & 0 & 0 \\
            -c & d & 0 & 0 \\
        \end{block}
        \begin{block}{rr(rr)}
            0 & 0 & -a & b \\
            0 & 0 & c & -d \\
        \end{block}
    \end{blockarray}
\right] \]

\subsection{Fonts}
$ \mathnormal{Hello123World} $

$ \mathit{Hello123World} $

$ \mathrm{Hello123World} $

$ \mathbf{Hello123World} $

$ \mathsf{Hello123World} $

$ \mathtt{Hello123World} $

$ \mathcal{Hello123World} $     % only support uppercase

$ \mathbb{Hello123World} $      % must usepackage amssymb, only support uppercase

$ \mathfrak{Hello123World} $    % must usepackage amssymb

Special symbol should use special font in math mode instead of by default: 
$ \mi $, $ \me $, $ \uppi $, $ \int f(x) \diff x $, $ \iint \frac{\dif y}{f(x, y)} \dif x $     % \uppi must usepackage upgreek

Multiply: $xyz$; Variable name: $ \mathit{xyz} $.

Bold math: 
{\boldmath{$a^2$}},                                 % vanilla latex
$ \boldsymbol{v} $, $ \pmb\sum $,                   % must usepackage amsmath
$ \bm{u} + \hm{u} $, $ \bm{\int} < \hm{\int} $      % must usepackage bm

Font size: 
\[\displaystyle
    \sum A(n)
\]
\[\textstyle
    \sum B(n)
\]
\[\scriptstyle
    \sum C(n)
\]
\[\scriptscriptstyle
    \sum D(n)
\]

\subsection{Symbols}
% TODO: extract commonly used symbols out
\subsubsection{\textcircled{1} Normal Symbols And \textcircled{2} Variable Family}
\paragraph{Lowercase Greek Letters}
{\LARGE
    $ \alpha, \beta, \gamma, \delta, (\epsilon), \zeta, \eta, \theta, \iota, \kappa, \lambda, \mu, 
    \nu, \xi, \pi, \rho, \sigma, \tau, \upsilon, (\phi), \chi, \psi, \omega $

    $ (\varepsilon), \vartheta, \varpi, \varrho, \varsigma, (\varphi), 
    \varkappa, \digamma $   % must usepackage amssymb (last two)
}

\paragraph{Uppercase Greek Letters}
{\LARGE
    $ \Gamma, \Delta, \Theta, \Lambda, \Xi, \Pi, \Sigma, \Upsilon, \Phi, \Psi, \Omega $

    $ \varGamma, \varDelta, \varTheta, \varLambda, \varXi, \varPi, \varSigma, \varUpsilon, \varPhi, \varPsi, \varOmega $    % must usepackge amsmath
}

\paragraph{Hebrew Alphabet}
{\LARGE
    $ \aleph $

    $ \beth, \daleth, \gimel $  % must usepackage amssymb
}

\paragraph{Upright Greek Letters}
% must usepackage upgreek
{\LARGE
    $ \upalpha, \upbeta, \upgamma, \updelta, \upepsilon, \upeta, \uptheta, \upiota, \upkappa, \uplambda, \upmu, 
    \upnu, \upxi, \uppi, \uprho, \upsigma, \uptau, \upupsilon, \upphi, \upchi, \uppsi, \upomega $

    $ \upvarepsilon, \upvartheta, \upvarpi, \upvarrho, \upvarsigma, \upvarphi $     % no \upvarkappa and \updigamma
}

\paragraph{Math Accents}
{\LARGE
    $ \bar{a}, \acute{a}, \grave{a}, \check{a}, \breve{a}, \tilde{a}, \hat{a}, \vec{a}, \mathring{a}, \dot{a}, \ddot{a}, 
    \ddddot{a}, \ddddot{a} $    % must usepackage amsmath (last two)

    $ \widetilde{abc}, \widehat{abc}, 
    \widetriangle{abc}, \wideparen{abc} $       % must usepackage yhmath

    $ abc\spbreve, abc\spcheck, abc\spdot, abc\spddot, abc\spdddot, abc\sphat, abc\sptilde $    % must usepackage amsxtra

    $ \accentset{*}{a}, \underaccent{*}{a}, \underaccent{\check}{a} $    % must usepackage accents
}

Change the accents position: $ \ddot{h} $ vs $ \skew{-2}{\ddot}{h} $.

\paragraph{Symbols}
{\LARGE
    $ \hbar, \hslash, \imath, \jmath, \ell, \wp, \Re, \Im, \mho, \eth, \Finv, \Game, \Bbbk $
    
    $ \partial, \infty, \prime, \backprime, (\emptyset), (\varnothing), \forall, \exists, \nexists, 
    \neg, \top, \bot, \surd, \angle, \sphericalangle, \measuredangle $
    
    $ \flat, \natural, \sharp $
    
    $ \nabla, \triangle, \vartriangle, \blacktriangle, \triangledown, \blacktriangledown, \square, \blacksquare, \lozenge, \blacklozenge, 
    \clubsuit, \spadesuit, \diamondsuit, \heartsuit, \bigstar $

    $ \backslash, \diagup, \diagdown, \circledS, \complement $
}

\subsubsection{\textcircled{3} Operators}
\paragraph{Large Operators}
\[
    \sum, \prod, \coprod, \int, \oint
\]
\[
    \bigcup, \biguplus, \bigsqcup, \bigvee, \bigwedge, \bigcap
\]
\[
    \bigodot, \bigoplus, \bigotimes
\]
\[
    \iint, \iiint, \iiiint, \idotsint       % must usepackage amsmath
\]

\paragraph{Word Operators}
\subparagraph{Non-limit}
{\LARGE
    $ \log x, \lg x, \ln x $
    
    $ \sin x, \arcsin x, \cos x, \arccos x, \tan x, \arctan x, \cot x $
    
    $ \sinh x, \cosh x, \tanh x, \coth x, \sec x, \csc x $

    $ \arg x, \ker x, \dim x, \hom x, \exp x, \deg x $

    $ r = m \bmod n $, $ r = m \pmod n $

    $ r = m \mod n $, $ r = m \pod n $      % must usepackage amsmath
}

\subparagraph{With-limit}
{\LARGE
    $ \lim x, \limsup x, \liminf x, \max x, \min x $
    
    $ \sup x, \inf x, \det x, \Pr x, \gcd x $

    $ \varliminf x, \varlimsup x, \injlim x, \projlim x, \varinjlim x, \varprojlim x $      % must usepackage amsmath
}

\paragraph{Customize Operators}
% must usepackage amsmath
{\LARGE
    $ |A| = \card (A) $     % must DeclareMathOperator

    $ \operatorname*{Prob}_{} (X) = 0 $     % if `*` then with limit
}

\subsubsection{\textcircled{4} Binary Operators And \textcircled{5} Relation Operators}
\paragraph{Binary Operators}
{\LARGE
    $ a + b, a - b, a * b $     % not include `/` (just normal symbol)

    $ a \triangleleft b, a \triangleright b, a \bigtriangleup b, a \bigtriangledown b $

    $ a \wedge b,   % also as a \land b
    a \vee b,       % also as a \lor b
    a \cap b, a \cup b, a \sqcap b, a \sqcup b, a \uplus b, a \amalg b $

    $ a \div b, a \ast b,      % the same as `*`
    a \times b, a \cdot b, a \star b, a \mp b, a \pm b, a \setminus b $

    $ a \circ b, a \bigcirc b, a \bullet b, a \diamond b, a \odot b, a \oslash b, a \otimes b, a \oplus b $

    $ a \wr b, a \dagger b, a \ddagger b $
}

\vspace{2em}

{\LARGE
% must usepackage amsmath
    $ a \dotplus b, a \smallsetminus b, a \intercal b, a \curlywedge b, a \curlyvee b, a \centerdot b $

    $ a \Cap b,     % also as a \doublecap b
    a \Cup b,       % also as a \doublecup b
    a \barwedge b, a \veebar b, a \doublebarwedge b $

    $ a \boxminus b, a \boxdot b, a \boxplus b, a \circleddash b, a \circledast b, a \circledcirc b $

    $ a \ltimes b, a \rtimes b, a \divideontimes b, a \leftthreetimes b, a \rightthreetimes b $

    $ a \lhd b, a \unlhd b, a \rhd b, a \unrhd b $
}

Used to turn into binary operator, $ a \mathbin{\heartsuit} b $.

\paragraph{Relation Operators}
{\LARGE
% not form always need usepackage amsmath
    $ a = b, a > b, a < b, a : b $

    $ a \neq b,     % also as a \ne b, no need for usepackage amsmath
    a \ngtr b,      % must usepackage amsmath
    a \nless b $    % must usepackage amsmath

    $ a \leq b,     % also as a \le b
    a \nleq b, a \lneq b, a \geq b,    % also as a \ge b
    a \ngeq b, a \gneq b, a \ll b,     % no not form
    a \gg b $       % no not form

    $ a \prec b, a \nprec b, a \succ b, a \nsucc b $

    $ a \preceq b, a \npreceq b, a \precneqq b, a \succeq b, a \nsucceq b, a \succneqq b $

    $ a \sim b, a \nsim b, a \approx b,     % no not form
    a \simeq b,     % no not form
    a \cong b, a \ncong b, a \equiv b,      % no not form
    a \doteq b $    % no not form

    $ a \in b, a \notin b,  % no need for usepackage amsmath
    a \ni b,        % also as a \owns b, no not form
    a \subset b,    % no not form
    a \supset b $   % no not form

    $ a \subseteq b, a \nsubseteq b, a \subsetneq b, a \varsubsetneq b, 
    a \supseteq b, a \nsupseteq b, a \supsetneq b, a \varsupsetneq b $

    $ a \smile b,   % no not form
    a \frown b,     % no not form
    a \perp b,      % no not form
    a \models b,    % no not form
    a \vdash b, a \nvdash b, a \dashv b $   % no not form

    $ a \mid b, a \nmid b, a \parallel b, a \nparallel b $
    

    $ a \propto b,  % no not form
    a \asymp b,     % no not form
    a \bowtie b, a \Join b $    % no not form
}

\vspace{2em}

{\LARGE
% must usepackage amsmath
    $ a \leqq b, a \nleqq b, a \lneqq b, a \lvertneqq b, 
    a \geqq b, a \ngeqq b, a \gneqq b, a \gvertneqq b $

    $ a \leqslant b, a \nleqslant b, a \geqslant b, a \ngeqslant b $

    $ a \lesssim b, a \lnsim b, a \gtrsim b, a \gnsim b, 
    a \lessapprox b, a \lnapprox b, a \gtrapprox b, a \gnapprox b $

    $ a \precsim b, a \precnsim b, a \succsim b, a \succnsim b, 
    a \precapprox b, a \precnapprox b, a \succapprox b, a \succnapprox b $

    $ a \subseteqq b, a \nsubseteqq b, a \subsetneqq b, a \varsubsetneqq b, 
    a \supseteqq b, a \nsupseteqq b, a \supsetneqq b, a \varsupsetneqq b $

    $ a \vartriangleleft b, a \ntriangleleft, a \vartriangleright b, a \ntriangleright b, 
    a \trianglelefteq b, a \ntrianglelefteq b, a \trianglerighteq b, a \ntrianglerighteq $

    $ a \shortmid b, a \nshortmid b, a \shortparallel b, a \nshortparallel b $

    $ a \vDash b, a \nvDash b, a \Vdash b, a \nVdash b, a \Vvdash b, a \nVDash $
}

\vspace{2em}

{\LARGE
% must usepackage amsmath, no not form
    $ a \eqslantless b, a \eqslantgtr b, a \approxeq b, a \lessdot b, a \gtrdot b, a \lll b, a \ggg b $

    $ a \lessgtr b, a \gtrless b, a \lesseqgtr b, a \gtreqless b, a \lesseqqgtr b, a \gtreqqless b $

    $ a \doteqdot b, a \triangleq b, a \eqcirc b, a \circeq b, a \risingdotseq b, a \fallingdotseq b $

    $ a \backsim b, a \thicksim b, a \backsimeq b, a \thickapprox b $

    $ a \preccurlyeq b, a \succcurlyeq b, a \sqsubseteq b, a \sqsupseteq b, 
    a \sqsubset b, a \sqsupset b, a \Subset b, a \Supset b $

    $ a \smallsmile b, a \smallfrown b, a \bumpeq b, a \Bumpeq b, 
    a \between b, a \pitchfork b, a \varpropto b, a \backepsilon b $

    $ a \blacktriangleleft b, a \blacktriangleright b $

    $ a \therefore b, a \because b $
}

Used to turn into relation operator, $ a \mathrel{\overline{\in}} b $.

\paragraph{Arrows}
{\LARGE
% not form always need usepackage amsmath
    $ a \leftarrow b,   % also as a \gets b
    a \nleftarrow b, a \rightarrow b,   % also as a \to b
    a \nrightarrow b $
    
    $ a \Leftarrow b, a \nLeftarrow b, a \Rightarrow b, a \nRightarrow b $

    $ a \leftrightarrow b, a \nleftrightarrow b, a \Leftrightarrow b, a \nLeftrightarrow b $

    $ a \longleftarrow b, a \longrightarrow b, a \Longleftarrow b, a \Longrightarrow b, a \longleftrightarrow b, a \Longleftrightarrow b $

    $ a \mapsto b, a \longmapsto b, a \hookleftarrow b, a \hookrightarrow b $

    $ a \leftharpoonup b, a \rightharpoonup b, a \leftharpoondown b, a \rightharpoondown b, a \rightleftharpoons b $

    $ a \nearrow b, a \searrow b, a \swarrow b, a \nwarrow b $

    $ a \uparrow b, a \Uparrow b, a \downarrow b, a \Downarrow b, a \updownarrow b, a \Updownarrow b $
}

\vspace{2em}

{\LARGE
% must usepackage amsmath
    $ a \leftleftarrows b, a \rightrightarrows b, a \leftrightarrows b, a \rightleftarrows b, a \upuparrows b, a \downdownarrows b $

    $ a \Lleftarrow b, a \Rrightarrow b, a \twoheadleftarrow b, a \twoheadrightarrow b, 
    a \leftarrowtail b, a \rightarrowtail b $

    $ a \leftrightharpoons b, a \rightleftharpoons b,   % redefine in amsmath
    a \upharpoonleft b, a \upharpoonright b,    % also as a \restriction b
    a \downharpoonleft b, a \downharpoonright b $

    $ a \curvearrowleft b, a \curvearrowright b, a \circlearrowleft b, a \circlearrowright b, 
    a \Lsh b, a \Rsh b, a \looparrowleft b, a \looparrowright b $

    $ a \multimap b, a \rightsquigarrow b, a \leftrightsquigarrow b, a \leadsto b $
}

\paragraph{Logical Operators}
$ P \iff Q $

$ P \implies Q $

$ P \impliedby Q $

\paragraph{Stack Relation}
{\LARGE
    $ f(x) \defeq ax+b $
}

\vspace{2em}

{\LARGE
    $ A \xleftarrow{0<x<1} B \xrightarrow[x\leq 0]{x\geq 1} C $   % must usepackage amsmath
}

\vspace{2em}

{\LARGE
% must usepackage extarrows
    $ A \xlongleftarrow{0<x<1} B \xlongrightarrow[x\leq 0]{x\geq 1} C $

    $ A \xLongleftarrow{0<x<1} B \xLongrightarrow[x\leq 0]{x\geq 1} C $

    $ A \xleftrightarrow[xyz]{a+b+c} B \xLeftrightarrow[xyz]{a+b+c} A 
    \xlongleftrightarrow[xyz]{a+b+c} B \xLongleftrightarrow[xyz]{a+b+c} A $

    $ A \xlongequal[xyz]{a+b+c} B $
}

\subsubsection{\textcircled{6} Parentheses And \textcircled{7} Delimiters}
$ (Hello) $, $ [Hello] $, $ \{ Hello \} $,      % also as \lbrace and \rbrace
$ \langle Hello \rangle $, $ \lfloor Hello \rfloor $, $ \lceil Hello \rceil $

$ \left( Hello \middle\vert World \right) $, $ \bigl( Hello \bigm\vert World \bigr) $, 
$ \Bigl( Hello \Bigm\vert World \Bigr) $, $ \biggl( Hello \biggm\vert World \biggr) $, 
$ \Biggl( Hello \Biggm\vert World \Biggr) $

\subsubsection{\textcircled{8} Punctuations}
$ x \colon f(x) $ vs $ x : f(x) $

$ 1, \ldots, n $, $ 1 + \cdots + n $, 
$ (1, \dots, 1) + \dots + (n, \dots, n) $   % must usepackage amsmath

$ 1, \vdots, n $, $ 1, \ddots n $, $ 1, \iddots n $     % \iddots must usepackage mathdots

$ 1, \dotsc, n $, $ 1 + \dotsb + n $, $ 1 \dotsm n $, $ \int_0^1 \dotsi \int_0^1 $, $ \dotso $  % must usepackage amsmath

\subsection{Multiple-line Formula}
\subsubsection{Different Formula Different Lines}
\begin{gather}
    a + b = b + a \\
    a - b = b - a \\
    a \times b = b \times a \notag \\
    a \div b \neq b \div a
\end{gather}    % if `*` then no index

\begin{align}
    d &= a + b + c & t &= r \times s \notag \\
    z &= x - y & q &= l \div m \div n \div o \div p
\end{align}     % if `*` then no index
\begin{align}
        & (a + b)(a - b) + 2ab \notag \\
    ={} & a^2 - b^2 + 2ab \notag \\
    ={} & (a - b)^2
\end{align}
\begin{align}
    &\mathrel{\phantom{=}} (a + b)(a - b) + 2ab \notag \\
    &= a^2 - b^2 + 2ab \notag \\
    &= (a - b)^2
\end{align}
\begin{align}
    x^2 + 2x &= -1
    \intertext{transpose got}
    x^2 + 2x + 1 &= 0
    \shortintertext{merge got}  % must usepackage mathtools
    (x + 1)^2 &= 0
\end{align}

\begin{flalign}
    2 &= 1 + 1      & 1 &= 1 \times 1 \\
    20 &= 10 + 10   & 100 &= 10 \times 10
\end{flalign}   % if `*` then no index

\begin{alignat}{2}
    x &= \sin t &\quad & y = \cos t \\
    z &= \tan t &      & w = \cot t
\end{alignat}   % if `*` then no index

\begin{subequations}
    \begin{align}
        a_{11}x + a_{12}y + a_{13}z &= A \\
        a_{21}x + a_{22}y + a_{23}z &= B \\
        a_{31}x + a_{32}y + a_{33}z &= C
    \end{align}
\end{subequations}

\subsubsection{One Formula Into Lines}
\begin{multline}
    1 + 2 + 3 \\
    \shoveleft{+ 4 + 5 + 6} \\
    + 7 + 8 + 9 \\
    \shoveright{+ 10 + 11 + 12} \\
    + 13 + 14 + 15
\end{multline}  % if `*` then no index

% \begin{equation}
%     \begin{split}
%         \cos 2x &= \cos^2 x - \sin^2 x \\
%                 &= 2\cos^2 x - 1
%     \end{split}                            % bug exists
% \end{equation}

% \begin{dmath}
%     \frac{1}{2} (\sin (x + y) + \sin (x - y)) = 
%     \frac{1}{2} (\sin x \cos y + \cos x \sin y) + \frac{1}{2} (\sin x \cos y - \cos x \sin y) = 
%     \sin x \cos y
% \end{dmath}     % must usepackage breqn, if `*` then no index, bug exists

\subsubsection{Lines Into One Formula}
\begin{equation}
    D(x) = 
    \begin{cases}
        1, & \text{if } x \in \mathbb{Q}; \\
        0, & \text{if } x \in \mathbb{R}\setminus\mathbb{Q}.
    \end{cases}
\end{equation}

\begin{numcases}{D(x)=}
    1, & if $x \in \mathbb{Q}$; \\
    0, & if $x \in \mathbb{R}\setminus\mathbb{Q}$.
\end{numcases}      % must usepackage cases

\begin{equation}
    \begin{gathered}
        S \subseteq T \\
        S \supseteq T
    \end{gathered}
    \implies S = T
\end{equation}      % by default centering, if `t` then align by the first line, if `b` then align by the last line
\begin{equation}
    \begin{aligned}
        f(x) &\leq x \\
        x    &\leq f(x)
    \end{aligned}
    \implies f(x) = x
\end{equation}      % by default centering, if `t` then align by the first line, if `b` then align by the last line
\begin{equation}
    \begin{alignedat}{2}
        f(x) &\in x &\quad g(x) &\subset x \\
        x    &\in f(x) &\quad x &\subset g(x)
    \end{alignedat}
    \implies f(x) = x, g(x) = x
\end{equation}      % by default centering, if `t` then align by the first line, if `b` then align by the last line

\subsection{Index}
\begin{equation}
    a^2 + b^2 = c^2 \notag
\end{equation}
\begin{equation*}
    a^2 + b^2 = c^2 \tag{$\star $}
\end{equation*}
\begin{equation*}
    a^2 + b^2 = c^2 \tag*{[gougu]}
\end{equation*}

\subsection{Seperate}
\[
    a + b \quad a {+} b \quad a \mathord{+} b
\]
\[
    \max n \quad {\max} n \quad \mathord{\max} n
\]

\begin{gather}
    [\,]\colon 3mu \\
    [\:] or [\>]\colon 4mu plus 2mu minus 4mu \\
    [\;]\colon 5mu plus 5mu \\
    [\!]\colon -3mu \\
    [\mspace{18mu}]\colon 18mu \\
    [\quad]\colon 1em \\
    [\qquad]\colon 2em
\end{gather}

\subsection{Hyphenation}
$ F(x)\* G(x)\* H(x) $
